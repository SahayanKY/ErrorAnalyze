\documentclass[10pt,dvipdfmx]{jsarticle}

%パッケージ
%
%
%-------------------------------パッケージ--------------------------------------
%
%
%--------------------------------数学関係---------------------------------------
	\usepackage{amssymb}
	\usepackage{amsmath}
	\usepackage{bm}

%--------------------------------graphicx---------------------------------------
	\usepackage[dvipdfmx]{graphicx}


%--------------------------------multirow---------------------------------------
	%表組みで縦セルを結合
	\usepackage{multirow}


%---------------------------------color-----------------------------------------
	\usepackage[dvipdfmx]{color}


%--------------------------------caption----------------------------------------
	%キャプションの体裁を調整
	\usepackage{caption}

	%表4.1「.」と最後にピリオドをつけて区切る
	\captionsetup{labelsep=period}

	%キャプション文を太字にし、一回り文字を小さくする
	\captionsetup{font = {bf,small}}

	%キャプション文を改行する際は、図X.Y.の分だけ文字を下げる
	\captionsetup{format = hang}

	%キャプション文全体の横幅を調整。大きくするほど横幅は小さくなる
	\captionsetup{margin = 60pt}


%----------------------------------tikz-----------------------------------------
	%tikz
	\usepackage{tikz}
	%矢印関連のライブラリ、ドキュメント16章
	\usetikzlibrary{arrows.meta}
	%\usetikzlibrary{calc,patterns,intersections}


%---------------------------------mhchem----------------------------------------
	%化学式記述用のパッケージ
	\usepackage[version=3]{mhchem}


%---------------------------------siunitx---------------------------------------
	%単位、数値フォーマット
	\usepackage{siunitx}

	%物理量のフォーマット
	%数値と単位の間を半角スペースに設定
	\sisetup{number-unit-product=\ }

	%誤差表示のフォーマット
	%有効にすると誤差を±で表示
	%\sisetup{separate-uncertainty}

	%表組において、S指定のカラムで文字列を中央に寄せる
	\sisetup{table-number-alignment = center}


%-----------------------------------url-----------------------------------------
	%参考文献でurlを正しく表示するのに使う
	\usepackage{url}


%マクロ
%
%
%-----------------------------数学関係のマクロ----------------------------------
%
%
%微小量
\renewcommand{\d}{{\ensuremath{\rm d}}}


%偏微分
\newcommand{\pard}[3]{{%
	\ifnum #2<1%
		\text{エラー}%
	\else{%
		\ifnum #2=1%
			\ensuremath{%
				\frac{\partial #3}{\partial #1}%
				%\partial_{#1} #3%
			}%
		\else{%
			\ensuremath{%
				\frac{\partial^{#2} #3}{\partial #1^{#2}}%
				%\partial_{#1}^{#2} #3
			}%
		}\fi%
	}\fi%
}}

%回転
\newcommand{\rot}{{\ensuremath{\nabla\times}}}


%発散
\renewcommand{\div}{{\ensuremath{\nabla\cdot}}}


%勾配
\newcommand{\grad}{{\ensuremath{\nabla}}}


%ラプラシアン
\newcommand{\laplacian}{{\ensuremath{\nabla^2}}}


%基底ベクトル
\newcommand{\e}[1]{{\ensuremath{\bm{e}_{#1}}}}


%法線ベクトル
\newcommand{\n}{{\ensuremath{\bm{n}}}}


%接線ベクトル
\renewcommand{\t}{{\ensuremath{\bm{t}}}}


%面ベクトル、Poyntingベクトル
\renewcommand{\S}{{\ensuremath{\bm{S}}}}


%位置ベクトル
\renewcommand{\r}{{\ensuremath{\bm{r}}}}


%位置ベクトル2
\newcommand{\x}{{\ensuremath{\bm{x}}}}


%平均
\newcommand{\average}[1]{{\ensuremath{\langle#1\rangle}}}



%
%
%------------------------------数字の体裁関係のマクロ--------------------------------
%
%
%
%ローマ数字の出力に使う
\newcommand{\ONE}{{\rm I}}
\newcommand{\TWO}{{\rm I\hspace{-.1em}I}}
\newcommand{\THREE}{{\rm I\hspace{-.1em}I\hspace{-.1em}I}}
\newcommand{\FOUR}{{\rm I\hspace{-.1em}V}}
\newcommand{\FIVE}{{\rm V}}



%〇数字などに使う
\newcommand{\circled}[1]{{\textcircled{\scriptsize #1}}}





\begin{document}
\section{誤差の伝播}
\subsection{一般式}
$x_1,x_2,\cdots,x_n$が独立の変数であり、
$y$はそれらの関数であるとき、
\begin{align}
	y
&=
	f(x_1,x_2,\cdots,x_n)
\end{align}
である。

そして、それらの誤差を$\delta x_1,\delta x_2,\cdots,\delta x_n,\delta y$
とすると、
\begin{align}
	\delta y^2
&=
	\left(
		\pard{x_1}{1}{f}
			\delta x_1
	\right)^2
	+
	\left(
		\pard{x_2}{1}{f}
			\delta x_2
	\right)^2
	+
	\cdots
	+
	\left(
		\pard{x_n}{1}{f}
			\delta x_n
	\right)^2
\end{align}
となる。少し具体的な関数形で考えてみる。

\subsection{関数の積の誤差}
次のような従属変数の誤差を考える。
\begin{align}
	y
&=
	f(x) g(x)
\end{align}

このとき$\delta y$は
\begin{align}
	\delta y^2
&=
	\left(
		\left(
			f' g +f g'
		\right)\delta x
	\right)^2
\end{align}

\subsection{関数の積の誤差2}
次は
\begin{align}
	y
&=
	f(x_1) g(x_2)
\end{align}
を考える。

\begin{align}
	\delta y^2
&=
	\left(
		f' g \delta x_1
	\right)^2
	+
	\left(
		f g' \delta x_2
	\right)^2
\end{align}

\subsubsection{関数の積の誤差3}
次は
\begin{align}
	y
&=
	f(x_1) g(x_1,x_2)
\end{align}
を考える。

\begin{align}
	\delta y^2
&=
	\left(
		\left(
			f' g
			+
			f \pard{x_1}{1}{g}
		\right) \delta x_1
	\right)^2
	+
	\left(
		f \pard{x_2}{1}{g}
			\delta x_2
	\right)^2 \\
%
%
&=
	f'^2 \delta x_1^2 \cdot g^2
	+
	2 f' g f \pard{x_1}{1}{g} \delta x_1^2
	+
	f^2
		\left(
			\left(
				\pard{x_1}{1}{g} \delta x_1
			\right)^2
			+
			\left(
				\pard{x_2}{1}{g} \delta x_2
			\right)^2
		\right) \\
%
%
&=
	\delta f^2 \cdot g^2
	+
	f^2 \delta g^2
	+
\end{align}


\section{誤差の表記}


\end{document}