%
%
%-----------------------------数学関係のマクロ----------------------------------
%
%
%微小量
\renewcommand{\d}{{\ensuremath{\rm d}}}


%偏微分
\newcommand{\pard}[3]{{%
	\ifnum #2<1%
		\text{エラー}%
	\else{%
		\ifnum #2=1%
			\ensuremath{%
				\frac{\partial #3}{\partial #1}%
				%\partial_{#1} #3%
			}%
		\else{%
			\ensuremath{%
				\frac{\partial^{#2} #3}{\partial #1^{#2}}%
				%\partial_{#1}^{#2} #3
			}%
		}\fi%
	}\fi%
}}

%回転
\newcommand{\rot}{{\ensuremath{\nabla\times}}}


%発散
\renewcommand{\div}{{\ensuremath{\nabla\cdot}}}


%勾配
\newcommand{\grad}{{\ensuremath{\nabla}}}


%ラプラシアン
\newcommand{\laplacian}{{\ensuremath{\nabla^2}}}


%基底ベクトル
\newcommand{\e}[1]{{\ensuremath{\bm{e}_{#1}}}}


%法線ベクトル
\newcommand{\n}{{\ensuremath{\bm{n}}}}


%接線ベクトル
\renewcommand{\t}{{\ensuremath{\bm{t}}}}


%面ベクトル、Poyntingベクトル
\renewcommand{\S}{{\ensuremath{\bm{S}}}}


%位置ベクトル
\renewcommand{\r}{{\ensuremath{\bm{r}}}}


%位置ベクトル2
\newcommand{\x}{{\ensuremath{\bm{x}}}}


%平均
\newcommand{\average}[1]{{\ensuremath{\langle#1\rangle}}}



%
%
%------------------------------数字の体裁関係のマクロ--------------------------------
%
%
%
%ローマ数字の出力に使う
\newcommand{\ONE}{{\rm I}}
\newcommand{\TWO}{{\rm I\hspace{-.1em}I}}
\newcommand{\THREE}{{\rm I\hspace{-.1em}I\hspace{-.1em}I}}
\newcommand{\FOUR}{{\rm I\hspace{-.1em}V}}
\newcommand{\FIVE}{{\rm V}}



%〇数字などに使う
\newcommand{\circled}[1]{{\textcircled{\scriptsize #1}}}



